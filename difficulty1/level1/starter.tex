% Remark that <notes> will create a notes preamble...
\documentclass{report}


\input{preamble}
\input{macros}
\input{letterfonts}
\newtheorem*{remark*}{Example}

%%%%%%%%%%%%%%%%%%%%%%

% Misc
%%%%%%%%%%%%%%%%%%%%%%
\newtheorem{plain}{Symbol}
%%%%%%%%%%%%%%%%%%%%%%

\title{title}
\author{randyrtt}
\begin{document}
\begin{sloppypar}
  \section{Problem name and description}%
  \label{sec:name}
  \begin{enumerate}
    \item \textbf{Problem Name.} \href{https://crackmes.one/crackme/646627a933c5d439389131d9}{\texttt{level1}} by \texttt{sudo0x18}
    \item \textbf{Description.} This is a text-based program written in a C/C++ family language. Whenever you execute the program (called \texttt{level1}), you are prompted to put in a password, which presumably has two outcomes: success and failure.
  \end{enumerate}

  \section{General Approach}%
  \label{sec:General Approach}
  Given the fact that the program is CLI-based as well as
  written in a C-family language, I intuitively just want to disassemble
  the program using \texttt{gdb} and just walk through the call stack, since
  it seems like it's just a simple string comparison program. \\
  Here is just a list of what I did:
  \begin{enumerate}
    \item Obtained filetype for the binary as well as printable strings
          within the binary
    \item Ran the program through \texttt{gdb}, creating breakpoints
          at the password checking function.
    \item Analyzed the disassembled instructions.
  \end{enumerate}


  \section{Prerequisite Knowledge + Keywords}%
  \label{sec:Prerequisite Knowledge}
  \begin{enumerate}
    \item Basic knowledge of \texttt{asm} language and
    \item Basic commands such as \texttt{mov} , \texttt{cmp} , \texttt{j} , and \texttt{call}

    \item \textbf{Keywords.} String comparison
  \end{enumerate}

  \subsection{Tools Used:}%
  \label{subsec:label}
  \begin{enumerate}
    \item \textbf{Disassembler.} \texttt{gdb} (GNU Debugger) or some other program like Ghidra, IDA Pro, etc.
  \end{enumerate}
  \section{Walkthrough}%
  \label{sec:Walkthrough}
  \subsection{Step $-1$: Running Malware Analysis Commands}%
  \label{subsec:Step0}
  As per the advice of other users, it appears that principally it maybe helpful to run a compiled program through the following commands:
  \begin{enumerate}
    \item \texttt{md5sum}: Hashing command that executes that
          \textbf{Message Digest 5} hashing algorithm directly
          on the compiled program's binary, which will return some value. in
          the world of malware analysis, compare this hash with what
          the original program author deems as the original hash
          for the program. Different hashes $\Rightarrow$ tampering.
    \item \texttt{sha1sum}: Hashing command that executes the \textbf{Secure
          Hash 1} cryptographic algorithm. Same principle as \texttt{md5sum}.
  \end{enumerate}
  Nothing interesting there.

  \subsection{Step 0: Getting Metadata on File}%
  \label{subsec:label}
  Now, we can actually do helpful things. First thing that I want to do is
  checkout the \textbf{filetype} of the file, as well as look at any
  strings that are encoded within the file.
  \begin{enumerate}
    \item \textbf{Check filetype.} I used the \texttt{file} command in Linux
          and found that \texttt{level1} is a ELF 64-bit
          binary, which is pretty typical for all Linux
          executables.
    \item \textbf{Find printable strings within binary.} used the \texttt{strings} command to find all notable strings within the binary.
  \end{enumerate}
  Notable things that I got from this process are that
  I can can see not only the prompt for putting in password, but I also
  see the prompts for when I get the password correct or incorrect. Notable
  functions also included in the binary are \texttt{main}  and \texttt{checkPass}, which of course, likely refers to checking the password
  that I put in. \texttt{gdb} time.

  \subsection{Step 1: Running the program through \texttt{gdb}}%
  \label{subsec:label}
  My initial idea was to just run \texttt{gdb} and create a breakpoint at
  the function \texttt{checkPass} and see what happens whenever we type
  in an incorrect password.
  \\ \\ I used the following commands:
  \begin{center}
    \texttt{\$ gdb level1} \\
    \texttt{gdb> b checkPass} \\
    \texttt{gdb> run}
  \end{center}
  \subsection{Step 2: Running through instructions using \texttt{disassemble} and \texttt{ni}  }%
  \label{subsec:label}
  \\ Whenever I break in the \texttt{checkPass} function, I decide
  to step through the instructions by using the \texttt{ni} (next instruction)
  command. After stepping through the function quite a bit, I get the following
  intuitions:
  \begin{enumerate}
    \item Whenever \texttt{checkPass} is called, our input is presumably
          passed from \texttt{\%rax} into \texttt{\%rdi}
    \item We are checking characters of the string \textit{individually},
          since we can observe the usage of the \texttt{cmp}  and the \texttt{jne} commands.
    \item It makes sense that if the comparison doesn't work, that \texttt{checkPass} returns a \texttt{false}, which signals to \texttt{main}
          that an incorrect password was inputted.
  \end{enumerate}
  The following \texttt{cmp} statements are made, with their corresponding
  letter:
  \begin{center}
    \texttt{cmp AL, 0x73} $\Rightarrow$ s \\
    \texttt{cmp AL, 0x75} $\Rightarrow$ u \\
    \texttt{cmp AL, 0x64} $\Rightarrow$ d \\
    \texttt{cmp AL, 0x6f} $\Rightarrow$ o \\
    \texttt{cmp AL, 0x30} $\Rightarrow$ 0 \\
    \texttt{cmp AL, 0x78} $\Rightarrow$ x\\
    \texttt{cmp AL, 0x31} $\Rightarrow$ 1\\
    \texttt{cmp AL, 0x38} $\Rightarrow$ 8 \\
  \end{center}
  Of course, this means that the password must be \texttt{sudo0x18}, which
  is correct.



















\end{sloppypar}
\end{document}
